\documentclass[11pt]{scrartcl}
\usepackage[parfill]{parskip}
\usepackage{graphicx}
\usepackage{booktabs}
\usepackage{tabulary}
\usepackage{float}
\usepackage{hyperref}

\graphicspath{{../images/}}

\title{\textbf{Project Management}}
\subtitle{Final assignment\\
            Release of a FLOSS product by a SME}
\author{Ricardo Garc\'ia Fern\'andez}
\date{\today}

\begin{document}

\maketitle

\vfill

\begin{flushright}
    \copyright  2013 Ricardo Garc\'ia Fern\'andez - ricardogarfe [at] gmail [dot] com.

    This work is licensed under a Creative Commons 3.0 Unported License.
    To view a copy of this license visit:
 
    \url{http://creativecommons.org/licenses/by/3.0/legalcode}.
\end{flushright}

\begin{figure}[h]
    \begin{flushright}	
        \includegraphics{by}
        \label{fig:by}
    \end{flushright}
\end{figure}

\newpage

\tableofcontents

\newpage

\section{SME Introduction}
\label{sec:sme-introduction}

\par \emph{Code Garden} is our SME. We develop software products using quality patterns. Quality patterns are our main goal, applying Quality patterns to software development as is. plants and the code can not be kept alone, they can grow with life, diverted, or wither in a forgotten place. So plants like the code, need extra care, some gardeners, so they can grow and flourish. 

\par \emph{Technical Debt} is not a monster chasing us in every development sprint is another \emph{ROL} that we accept and we have interaction with it. We need him and he needs us.

\par Thus, we created a software product that helps us to deal with Technical Debt, \textbf{Greenhouse}.

\par \emph{Greenhouse} is our tool to track the progress of the evolution of code quality within a controlled environment. Using quality metrics for each programming language helps us reduce technical debt faster

\begin{quote}
    \begin{center}
    \emph{"we are the code we write"}
    \end{center}
\end{quote}

% section sme-introduction (end)
\section{Publish the project}
\label{sec:publish-project}

\par We want to publish \emph{Greenhouse} as a FLOSS\footnote{Free Libre Open Source Software} project with two Software Licenses.

\par One a FLOSS License and the other a Private License. We chose a \emph{Dual-License} product.

\par Thus we can serve every type of costumers as MySQL business model explained by Elena Blanco\cite{dl-business-model}.

\begin{quotation}
    \emph{For anyone who wants to develop and distribute but does not want to release the source code for their application, MySQL is able to provide a commercial licence. Because MySQL has full ownership of the MySQL code, it is able to tailor its commercial licensing terms to meet the unique requirements of users interested in embedding or bundling MySQL.}
\end{quotation}

\subsection{Dual License}
\label{sub:dual-license}

\par Free Software License and Private, Brief discussion about licenses (your company has heard about some BSD or GPL, but they are not sure!).

\par FLOSS License selected to publish \emph{Greenhouse} is \emph{GPLv3}\footnote{GNU Public License Version 3 - \url{http://www.gnu.org/licenses/gpl-3.0.html}}. This FLOSS License provides all freedoms to the project:

\begin{itemize}
	\item the freedom to use the software for any purpose,
	\item the freedom to change the software to suit your needs,
	\item the freedom to share the software with your friends and neighbors, and
	\item the freedom to share the changes you make.
\end{itemize}

\par The other License is a private License. A private License gives us more flexibility through market niche.

\par Good thing to take in advance using a Private License are:

\begin{itemize}
	\item Avoid possible or unexpected License Violations.
	\item Build a project in your company and be 100\% sure that you are not violation any License under your product.
	\item This License envelops the whole product into a private version.
\end{itemize}

\par This private License gives you the opportunity to deal with \emph{Greenhouse} libraries and include into other projects using full capabilities. This way you can avoid License problems with derivated works and be compatible with other FLOSS Licenses.

\par Brian Behlendorf\cite{brian-behlendorf-business-strategy} explains this model with a success and its weaknesses with the community development:

\begin{quotation}
    \emph{You have to be very careful, though, to make sure that any code volunteered to you by third parties is explicitly available for this non-free branch; you ensure this by either declaring that only you (or people employed by you) will write code for this project, or that (in addition) you'll get explicit clearance from each contributor to take whatever they contribute into a non-free version.}
\end{quotation}

\par We are very concerned about how to interact with the community. Our goal is provide to community the opportunity to develop solutions to our clients for \emph{Greenhouse}.

\begin{quotation}
    \emph{I would claim that if you treat your contributors right, perhaps even offer them money or other compensation for their contributions (it is, after all, helping your commercial bottom line), this model could work.}
\end{quotation}

\par Last quotation is how \emph{Transvirtual}\footnote{\url{http://www.transvirtualsystems.com/}} in Berkeley applied this model to a commercial lightweight Java. Encourage the developers to work and earn money directly with the software they make. We believe in this model to create a community involved around the product: 

\begin{itemize}
	\item The usability of the product.
	\item Tangible incentives as our case the money.
\end{itemize}

\par In this way we will try to solve the dilemma of FLOSS developments that can be converted into proprietary solutions.

\par Therefor, an exchange of knowledge by salaries to the community through an assignment of copyright to Code Garden by using the \emph{GPLv3 License}, they maintain the authoring and earn tangible rewards too.

\par We are the copyright holders of \emph{Greenhouse} and thus , we want to share the maintenability with community knowledge to provide solutions for companies with private Licenses of the product.

\par We want to evolve with the community and spread our developments and remain FLOSS.

% subsection dual-license (end)
% section publish-project (end)
\section{Market niche - Competitors analysis}
\label{sec:market-niche}

\par Code analysis is increasing everyday in software development. SME \& Big Enterprises focus its products near quality. Why ? Because Software Development is measurable, high measurable I could say. Every software is developed guided by patterns through developers and the final product (talking about clients) is released to the client showing its functionality but what happens with all the code developed inside ?

\par The code evolves as the development grows. In a development team is difficult to measure the quality of the product grained.

\par Other sample of the use of measure the quality is when you have to choose between two libraries to implement another service that use the functionality implemented by them. One metric to take care is the code quality that you can apply for them. Using some objective metrics you can retrieve a numeric result that gives you a general idea. Or if you want to add an existing module/library you can choose the one which its result is near to your software, not lesser not higher. It is a way of seeking for a balance in development and knowledge about complexity of a module to import to your product.

\par There are some samples of Analyzing tools but we are to analyze \emph{Code Climate} and \emph{Sonar}.

\subsection{Code climate}
\label{sub:code-climate}

\textbf{Code climate} - \url{https://codeclimate.com/}. Code quality analysis in Ruby Language. But we will not focus on language but with what gives us the tool.

\begin{figure}[H]
\centering
\includegraphics[width=0.7\textwidth]{code-climate.png}
\caption{Code Climate Notifications}
\label{}
\end{figure}

\par This tool gives us interoperability with Software Forges like Github integration. Source Code Management with Git repositories, Code Quality evolution, Notifications, Comparison tools. This tool set is simple but efficient, this is very important. Play with the ease of integration into a forge development offering hosting server, pay and enjoy the product (one click purchase).

\par Includes the main features of services; \emph{PaaS}\footnote{Platform as a Service}, \emph{IaaS}\footnote{Infrastructure as a Service} and \emph{SaaS}\footnote{Software as a Service}. Using \emph{bluebox services} - \url{https://bluebox.net/}: Virtualized Environments on Actual Hardware.

\par We want those services available for every developer and easily result visualization.

\par It's free (gratis) to use for FLOSS projects.

% subsection code-climate (end)

\subsection{Sonar}
\label{sub:sonar}

\textbf{Sonar} - \url{http://www.sonarsource.org/}. Sonar slogan is very clear:

\begin{quote}
    \emph{Put your technical debt under control}
\end{quote}

\par Despite Code Climate, Sonar covers lots of languages and is a FLOSS product. A very good point to consider for the use of this software. Controls and test every corner of your project.

\begin{figure}[H]
\centering
\includegraphics[width=0.7\textwidth]{sonar7axes.png}
\caption{Sonar covers 7 axes}
\label{sonar7axes}
\end{figure}

\par This Software gives you the opportunity to integrate with your Source code management and a result metrics result visualization panel with Nemo - \url{http://nemo.sonarsource.org/} using Saas solution through \emph{Cloud Bees} - \url{http://www.cloudbees.com/platform-service-sonarsource.cb}.

\begin{figure}[H]
\centering
\includegraphics[width=0.7\textwidth]{nemo-evolution.png}
\caption{}
\label{}
\end{figure}

\par Sonar is focused in code Analysis its market niche as Code Climate but offering easy configurable solutions for integrate your projects with giving you the possibility to analyze our software and visualize the results among time.

\par This is our niche, Software Analysis, visualization and interoperability between Source Code Management.

% subsection sonar (end)
% section market-niche (end)

\section{Management general path}
\label{sec:management-path}

\par services and personal.

% section management-path (end)

\section{Management Policies}
\label{sec:management-policies}

\par Communities, Enterprise ROLE, Single Vendor or Apache Software Foundation.

% section management-policies (end)
\subsection{Where will be published the code ?}
\label{sub:publish-code}

% subsection publish-code (end)

\subsection{Communication strategy and channels}
\label{sub:communication-strategy}

\par Documentation, Netiquette.

% subsection communication-strategy (end)

\subsection{Managing volunteers and attracting new users}
\label{sub:volunteers-users}

% subsection volunteers-users (end)

\section{Technology}
\label{sec:technology}

\par Commodity.

% section technology (end)

\subsection{Technical Infrastructure needed}
\label{sub:infrastructure}

\par Rationality, critical analysis, Development plan (good practices for source code development) and roadmap.

% subsection infrastructure (end)

\section{Business scalability}
\label{sec:scalability}

\par Metasploit and MySQL.

% section scalability (end)

\subsection{Evolution}
\label{sub:evolution}

\par Teams, Volunteers, Expansion. Where , How, Which mechanisms ?

% subsection evolution (end)

\subsection{Emphasis}
\label{sub:emphasis}

\par Integration, Upstream.

% subsection emphasis (end)

\begin{thebibliography}{9}

    \bibitem{your-cake}
    Philip H. Albert,\\
    Dual Licensing: Having Your Cake and Eating It Too,\\
    http://www.linuxinsider.com/story/38172.html

    \bibitem{revoking-gpl}
    Milking The GNU,\\
    Dual-licensing: revoking the GPL,\\
    http://blog.milkingthegnu.org/2008/05/dual-licensing.html

    \bibitem{unfair}
    Milking The GNU,\\
    Dual-licensing is unfair and community debilitating,\\
    http://blog.milkingthegnu.org/2008/05/exisiting-dual.html

    \bibitem{mit-gpl}
    StackOverflow,\\
    MIT GPL Dual-license in commercial software,\\
    http://stackoverflow.com/questions/3336161/mit-gpl-dual-license-in-commercial-software

    \bibitem{dual-license-schemes}
    Producing OSS,\\
    Dual Licensing Schemes,\\
    http://producingoss.com/en/dual-licensing.html

    \bibitem{dl-business-model}
    Elena Blanco,\\
    Dual-Licensing As A Business Model,\\
    http://www.oss-watch.ac.uk/resources/duallicence2

    \bibitem{brian-behlendorf-business-strategy}
    Brian Behlendorf,\\
    Open Source as a Business Strategy,\\
    http://oreilly.com/openbook/opensources/book/brian.html
\end{thebibliography}

\end{document}
