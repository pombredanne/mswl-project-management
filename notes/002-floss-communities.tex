\documentclass[11pt]{scrartcl}
\usepackage[parfill]{parskip}
\usepackage{graphicx}
\usepackage{booktabs}
\usepackage{tabulary}
\usepackage{float}
\usepackage{eurosym}
\usepackage{hyperref}

\graphicspath{{../images/}}

\title{\textbf{Project Management}}
\subtitle{FLOSS Communities}
\author{Ricardo Garc\'ia Fern\'andez}
\date{\today}

\begin{document}

\maketitle

\vfill

\begin{flushright}
    \copyright  2013 Ricardo Garc\'ia Fern\'andez - ricardogarfe [at] gmail [dot] com.

    This work is licensed under a Creative Commons 3.0 Unported License.
    To view a copy of this license visit:
 
    \url{http://creativecommons.org/licenses/by/3.0/legalcode}.
\end{flushright}

\begin{figure}[h]
    \begin{flushright}	
        \includegraphics{by}
        \label{fig:by}
    \end{flushright}
\end{figure}

\newpage

\section{FLOSS Community}

\begin{itemize}
	\item Increasing complexity
	\item Increasing commercial interest
	\item Relations between communities and corporations
	\item There is a need to bring the two worlds together
\end{itemize}

\par Dedicated new figure: The Community Manager

% section (end)
\section{Importance of Communities}
\label{sec:importance}

\begin{itemize}
	\item Larry Lessig Read/Write one\footnote{\url{http://www.ted.com/speakers/larry_lessig.html}}.
	\item Architecture of participation: nature of system that are designed for user contribution - Tim O'Reilly.
\end{itemize}

\par \emph{Downstream/Upstream} in FLOSS.

\par \emph{Downstream}, use the FLOSS tool, use the last features done by the community. Evolve the tool only for you, not directly for the community and take profit of other improvements.

\par \emph{Upstream}, are the improvement done by the user (you) that are pushed to the community.

\par Evolve in Community doing a good use of the upstream.

% section (end)
\section{What is a community manager?}
\label{sec:what-is}

\par Que cualidades tiene que tener un community manager ?
\begin{itemize}
	\item Know technical features of the project. Ha de conocer los aspectos técnicos del proyecto
	\item Public relationships. Transmitir la empatía por el proyecto.
	\item Mediation, diplomacy, listen and understand. Sabe detectar los problemas de la gente que trabaja en el proyecto, incluso anticipar los problemas.
\end{itemize}

\par Seth Godin\footnote{\url{http://sethgodin.typepad.com/}}: The online community organizer (or community manager) is one of the most important jobs of the future.

\par Es, al fin y al cabo, una persona que tiene que enlazar y mediar entre los distintos actores de la comunidad.

\subsection{Building community}
\label{sec:build-com}

% TBC
\begin{itemize}
	\item Face to face meetings
	\item Stands at conferences
	\item Online training events
	\item Bug squashing parties
	\item Newsletter, planet of blogs
\end{itemize}

% subsection (end)

\subsection{How to hire a Community manager}
\label{sub:}

Como valorar al community manager ?

\begin{itemize}
	\item Quantitativa: calidad de la comunidad, trending-topics, listas de correo, comunicación, consultar otras comunidades.
	\item Qualitativa: Mejora la calidad de la comunidad.
\end{itemize}

% subsection (end)
% section (end)

\section{Outreach strategy in FLOSS projects}
\label{sec:outreach}

Writing clearly is sometimes more important than programming skills.
    Empathy.
    Listening to othes.
    Explaining one’s point of view.
    Use different channels.
Different levels.
    Developer-developer.
    Developer-user.
    Developer-manager (and vice-versa).
    Etc.

\section{Control your public image.}
\label{sec:}

\par \textbf{Pensar antes de escribir}, ésta debe ser la primera regla para la buena gestión de la comunidad. 

\par Los mensajes han de ser constructivos, no incentivar la polémica (flame) e intentar no convertirse en un foco de problemas para la comunidad. 

\par La comunidad no quiere tener cerca a gente problemática, es decir, termina obviando a los ciudadanos de la comunidad que generan problemas. Si el interés en el proyecto decae debido a los malos modos en la comunidad, lo mejor es irse.

\par Se ha de mantener un buena ambiente en la comunidad. Cuando la gestión de los roles de la comunidad está descentralizada, los problemas son ínfimos. Si se encuentra un problema, es más rápido acabar con el.

\section{Common pitfails}
\label{sec:}

\emph{Noisy minority}: se escuche más su voz por insistencia, es decir por repetición, porque se conoce, estar en todos los lados y no por aportes técnicos.

\par Comunidades Asamblearias, puramente asamblearias, el 99\% falla, aunque se ha de buscar el consejo asambleario mediante representantes.

% section  (end)

\subsection{Security issues}
\label{sec:sec-issues}

Mantener informados a los usuarios de la herramienta.

% section  (end)

\subsection{Branding}
\label{sub:branding}

Identificar a tu proyecto con una idea. Demostrar las cualidades asociadas al proyecto asociandolo a un nombre.

% subsection  (end)

\section{Activity}
\label{sec:activity}

Nettiquette 

\begin{itemize}
	\item Mailing Lists
	\item IM, irc, whatsup
	\item Forums, wikis
\end{itemize}

\subsection{IM, irc, whatsup synchronous communication}
\label{sub:im}

Real time communication best practices.
Quick answer, availability, broadcasting ? 

% subsection im (end)
% section (end)
\end{document}
