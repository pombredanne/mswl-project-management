\documentclass[11pt]{scrartcl}
\usepackage[parfill]{parskip}
\usepackage{graphicx}
\usepackage{booktabs}
\usepackage{tabulary}
\usepackage{float}
\usepackage{eurosym}
\usepackage{hyperref}

\graphicspath{{../images/}}

\title{\textbf{Project Management}}
\subtitle{Volunteers management}
\author{Ricardo Garc\'ia Fern\'andez}
\date{\today}

\begin{document}

\maketitle

\vfill

\begin{flushright}
    \copyright  2013 Ricardo Garc\'ia Fern\'andez - ricardogarfe [at] gmail [dot] com.

    This work is licensed under a Creative Commons 3.0 Unported License.
    To view a copy of this license visit:
 
    \url{http://creativecommons.org/licenses/by/3.0/legalcode}.
\end{flushright}

\begin{figure}[h]
    \begin{flushright}	
        \includegraphics{by}
        \label{fig:by}
    \end{flushright}
\end{figure}

\newpage

\section{So you want to manage an open-source project\ldots}

\emph{Matt Asay}: So you want to manage an open-source project\ldots\footnote{\url{http://news.cnet.com/8301-13505_3-9821100-16.html}}.

\par Ambientado en 2007 cuando el auge del OpenSource estaba en la cúspide. La consultora Garnter planteaba el giro empresarial contando con este sector, la empresa que lo obviara estaría fuera de todo mercado.

\par Debate about the main topics in the article.

\begin{quote}
    \emph{The answer is no. The researchers find that the sheer amount of a person's technical contribution does not necessarily guarantee a position for him or her on the leadership team as project leader, project secretary, or developer account manager.}
\end{quote}

\begin{quote}
    \emph{Namely, those who care about a project as a community and nurture it, rather than those who simply write th best code within that community.}
\end{quote}

\emph{Meritocracy}: not only develop or be nice people. A mixture of both. Agreements, opinions, 

\section{Bikeshed}
\label{sec:bikeshed}

\begin{quote}
    \emph{Why Should I Care What Color the Bikeshed Is?}
\end{quote}

\emph{Poul-Henning Kamp} email about how to avoid stupid discussions about little changes\footnote{\url{http://bikeshed.org/}}.

\begin{itemize}
	\item Manage time and productivity.
	\item Newbies, are collapsed by barriers. Don't bite the newbies.
\end{itemize}

% section bikeshed (end)

\section{Anti-patterns}
\label{sec:anti-patterns}

\begin{quote}
    \emph{"Anti-patterns are damaging or counter-productive practices or patterns of behaviour that are sometimes seen in communities".}
\end{quote}

\par Wikia Anti-patterns\footnote{\url{http://communitymgt.wikia.com/wiki/Category:Anti-patterns}}

\begin{itemize}
	\item Help Vampire
	\item Cookie Licking
	\item Nepotism
	\item Decision paralysis
	\item The Big Show
	\item Me too
	\item I'm the bad guy?
	\item Anonymity
\end{itemize}

\par How to defeat them? \url{http://www.slideshare.net/nearyd/community-antipatterns#}

% section anti-patterns (end)
\end{document}
