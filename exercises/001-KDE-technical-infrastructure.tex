\documentclass[11pt]{scrartcl}
\usepackage[parfill]{parskip}
\usepackage{graphicx}
\usepackage{booktabs}
\usepackage{tabulary}
\usepackage{float}
\usepackage{eurosym}
\usepackage{hyperref}

\graphicspath{{../images/}}

\title{\textbf{Project Management}}
\subtitle{Technical infrastructure in KDE}
\author{Ricardo Garc\'ia Fern\'andez}
\date{\today}

\begin{document}

\maketitle

\vfill

\begin{flushright}
    \copyright  2013 Ricardo Garc\'ia Fern\'andez - ricardogarfe [at] gmail [dot] com.

    This work is licensed under a Creative Commons Attribution- ShareAlike 3.0 License.
    To view a copy of this license visit:
 
    \url{http://creativecommons.org/licenses/by-sa/3.0/legalcode}.
\end{flushright}

\begin{figure}[h]
    \begin{flushright}	
        \includegraphics{by}
        \label{fig:by-sa}
    \end{flushright}
\end{figure}

\newpage

\section{Questions}

The main website for the \emph{KDE} community is \url{http://www.kde.org/}. Visit this site and follow the links to answer these questions:

\begin{enumerate}
	\item What is the main document that we must read if we are interested in starting to contribute to a \emph{KDE} project, as a developer?
	\item What is the main aim of \emph{KDE TechBase}?
	\item Find out at least 5 different software tools that \emph{KDE} uses to manage their projects (including, for instance, version control systems, bug-tracking systems, communication, etc.). Include the URL to each tool to backup your answer.
\end{enumerate}

\end{document}
