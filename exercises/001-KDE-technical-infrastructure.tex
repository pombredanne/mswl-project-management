\documentclass[11pt]{scrartcl}
\usepackage[parfill]{parskip}
\usepackage{graphicx}
\usepackage{booktabs}
\usepackage{tabulary}
\usepackage{float}
\usepackage{eurosym}
\usepackage{hyperref}

\graphicspath{{../images/}}

\title{\textbf{Project Management}}
\subtitle{Technical infrastructure in KDE}
\author{Ricardo Garc\'ia Fern\'andez}
\date{\today}

\begin{document}

\maketitle

\vfill

\begin{flushright}
    \copyright  2013 Ricardo Garc\'ia Fern\'andez - ricardogarfe [at] gmail [dot] com.

    This work is licensed under a Creative Commons Attribution- ShareAlike 3.0 License.
    To view a copy of this license visit:
 
    \url{http://creativecommons.org/licenses/by-sa/3.0/legalcode}.
\end{flushright}

\begin{figure}[h]
    \begin{flushright}	
        \includegraphics{by}
        \label{fig:by-sa}
    \end{flushright}
\end{figure}

\newpage

\section{Questions}

The main website for the \emph{KDE} community is \url{http://www.kde.org/}. Visit this site and follow the links to answer these questions:

\begin{enumerate}
	\item What is the main document that we must read if we are interested in starting to contribute to a \emph{KDE} project, as a developer?
	\item What is the main aim of \emph{KDE TechBase}?
	\item Find out at least 5 different software tools that \emph{KDE} uses to manage their projects (including, for instance, version control systems, bug-tracking systems, communication, etc.). Include the URL to each tool to backup your answer.
\end{enumerate}

\section{KDE must read}
\label{sec:q-01}

\emph{What is the main document that we must read if we are interested in starting to contribute to a \emph{KDE} project, as a developer?}

\par First of all, look for the link regarding the \emph{KDE} development community on its website (\url{http://www.kde.org/}). We see highlights, among many options: Get Involved, Manifesto and Code of Conduct.

\begin{itemize}

	\item We should first read the manifesto, where they settle the ethical foundations of the project (\url{http://manifesto.kde.org/}).
	\item As a second article, we enter into the Code of Conduct (\url{http://www.kde.org/code-of-conduct/}). Which explains the internal rules of the community based on respect and communication.
	\item Last but not least, there is the section Get Involved (\url{http://community.kde.org/Getinvolved}). This section shows the gateway to the various forms of collaboration in the KDE community.
\end{itemize}

% section  (end)
\section{KDE TechBase}
\label{sec:q-02}

\emph{What is the main aim of \emph{KDE TechBase}?}

KDE TechBase\footnote{\url{http://techbase.kde.org/KDE_TechBase:About}} is a Wiki maintained by the KDE community. Wants to be the reference for the dissemination of knowledge about \emph{KDE}. A place where all types of volunteers contribute their knowledge of \emph{KDE} forming an encyclopedia of tutorials around \emph{KDE}. It consists of different types of tutorials for contributors (developers, translators, documentarians). Its growth is linked to the community. The more help the community, the better the community support.

% section  (end)

\section{KDE ALM Tools}
\label{sec:q-03}

\emph{Find out at least 5 different software tools that \emph{KDE} uses to manage their projects (including, for instance, version control systems, bug-tracking systems, communication, etc.). Include the URL to each tool to backup your answer.}

\emph{ALM Tools - Application lifecylce management Tools}: tools that facilitate and integrate requirements management, architecture, coding, testing, tracking, and release management\footnote{\url{http://en.wikipedia.org/wiki/Application_lifecycle_management\#Categories_of_ALM_tools}}.

\par Through TechBase can find different tools that shape the ALM ecosystem.

we have:

\begin{itemize}

	\item Communicating with the team - \#kde-devel IRC - irc://irc.kde.org/kde-devel
	\item Mailing List - \url{https://mail.kde.org/mailman/listinfo/kde-devel} and for anyone interested in other fields \url{http://www.kde.org/support/mailinglists/}.

	\item Reporting Bugs Bugzilla using KDE Bugtracking system in \url{https://bugs.kde.org/}.

	\item Getting and building the code - \url{http://techbase.kde.org/Getting_Started/Sources} and \url{http://api.kde.org/} for VCS Tools.

    \begin{itemize}
	    \item SVN - \url{http://techbase.kde.org/Getting_Started/Sources/Subversion}
	    \item Git - \url{http://techbase.kde.org/Development/Git/Recipes}
    \end{itemize}
    
	\item Community guides

    \begin{itemize}
        \item Policies - \url{http://techbase.kde.org/Policies}
     	\item Menthoring program - \url{http://community.kde.org/GSoC}
    \end{itemize}

	\item Developer Documentation - KDE TechBase \url{http://techbase.kde.org/}.
	\item Translation guides - Lokalize - KDE users guide in \url{http://userbase.kde.org/Lokalize}

\end{itemize}

% section  (end)
\end{document}
