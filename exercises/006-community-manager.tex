\documentclass[11pt]{scrartcl}
\usepackage[parfill]{parskip}
\usepackage{graphicx}
\usepackage{booktabs}
\usepackage{tabulary}
\usepackage{float}
\usepackage{eurosym}
\usepackage{hyperref}

\graphicspath{{../images/}}

\title{\textbf{Project Management}}
\subtitle{Art of Community Building}
\author{Ricardo Garc\'ia Fern\'andez}
\date{\today}

\begin{document}

\maketitle

\vfill

\begin{flushright}
    \copyright  2013 Ricardo Garc\'ia Fern\'andez - ricardogarfe [at] gmail [dot] com.

    This work is licensed under a Creative Commons 3.0 Unported License.
    To view a copy of this license visit:
 
    \url{http://creativecommons.org/licenses/by/3.0/legalcode}.
\end{flushright}

\begin{figure}[h]
    \begin{flushright}	
        \includegraphics{by}
        \label{fig:by}
    \end{flushright}
\end{figure}

\newpage

\section{Questions}

\par Read Chapter 13 of \emph{Jono Bacon}'s book "Art of Community Building" (\url{http://www.artofcommunityonline.org/get/}) and answer the following questions:

\begin{itemize}
	\item Which are the main skills/assets of a good community manager?
	\item Which are the main caveats to be considered when hiring a community manager?
\end{itemize}

\section{Which are the main skills/assets of a good community manager?}
\label{sec:main-skills}

\par After reading Chapter 13 of \emph{Jono Bacon}'s book "Art of Community Building". We can do a mixted definition of which should be main skills/assets of a community manager. 

\par Community managers must be open-minded people, good communicators, have knowledge of the technical side of the business (referred to the section of the workers) and outside users/customers of the product.

\begin{quote}
    \emph{"The ability to enact change, you should ensure that your community manager has a degree of control over changes and refinements to parts of the organization upon which the community’s work depends."}
\end{quote}

\par Easy speaking, understanding, innovation, reasoning and teamwork. That is, the community manager has to deal with different roles in the company but as an horizontal worker.

\par Community manager has to be a real person, generate empathy and interest, and be known as the person who is, integrate.

\begin{quote}
    \emph{"You should ensure that your hires are free enough to exercise individuality in this role."}
\end{quote}

\par Should know how to transmit information from each of the other levels between roles, ie, from the position of a chief, until the last worker who has started working in the company. Must be horizontal, neutral mediator.

\par Besides communication and mediator skills, community manager has to get involved in the community as another user. 

\begin{quote}
    \emph{"be a member of that community and exhibit the culture of that community"}
\end{quote}


% section main-skills (end)
\section{Which are the main caveats to be considered when hiring a community manager?}
\label{sec:main-caveats}

\par Jono Bacon definition of community:
\begin{quote}
    \emph{"Community is implicitly a positive word. It speaks of openness, participation, awareness, and an agreeable intention to engage in an environment driven by merit as well as caring for others."}
\end{quote}

\par When you have to hire a community manager, first of all you have to know your community and what are your expectations in a future. 

\par Knowing your community show you an idea of how it works, how is the communication and its members. Your company is a community itself.

\begin{itemize}
	\item Who is in your community?
	\item How big is it?
	\item What kinds of skills and diversity are present in your community?
	\item How does the community interact and work with your company?
	\item What kind of governance infrastructure is in place?
	\item Who are the contentious people, and what are the contentious topics?
\end{itemize}

\par What do you expect ?
\begin{itemize}
	\item How do you want to better understand who your community is?
	\item How would you like to grow the skills and diversity in your community? What are the primary skills and roles that you would like to focus on?
	\item How would you like to change, improve, and otherwise focus on how your community works with the company?
	\item What new and improved governance is required, and where should you focus your efforts first?
	\item How would you like to resolve and improve relations with the contentious people and topics in the community?
\end{itemize}

\par After this exercise with your community, you have to extrapolate your needs into a community manager. How a community manager could handle this aims.

\par The community manager must be prepared to answer questions on how to improve the community, as is his dealings with other communities, demonstrate their technical qualities, willingness to learn, interest, people skills, interoperability, conflict resolution, recommendations from other communities, if present in FLOSS communities, public profile, interests beyond work (can apply to another field), rationality, quantify the improvement of other communities where it has been, close presence and of course being a good communicator , receptor and mediator.

\par For this, the interviewer must know and keep abreast of these qualities, as well as a good communicator, mediator and be clear, why and why your community needs a community manager will be a solution.

\par \emph{Be clear that without teamwork with the community manager (at all levels) this role will not work. If there is no communication what the company needs is a change of attitude rather than a community manager.}

% section main-caveats (end)
\end{document}
