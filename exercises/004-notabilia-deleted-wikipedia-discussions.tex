\documentclass[11pt]{scrartcl}
\usepackage[parfill]{parskip}
\usepackage{graphicx}
\usepackage{booktabs}
\usepackage{tabulary}
\usepackage{float}
\usepackage{eurosym}
\usepackage{hyperref}

\graphicspath{{../images/}}

\title{\textbf{Project Management}}
\subtitle{Notabilia: Article for deletion}
\author{Ricardo Garc\'ia Fern\'andez}
\date{\today}

\begin{document}

\maketitle

\vfill

\begin{flushright}
    \copyright  2013 Ricardo Garc\'ia Fern\'andez - ricardogarfe [at] gmail [dot] com.

    This work is licensed under a Creative Commons Attribution- ShareAlike 3.0 License.
    To view a copy of this license visit:
 
    \url{http://creativecommons.org/licenses/by-sa/3.0/legalcode}.
\end{flushright}

\begin{figure}[h]
    \begin{flushright}	
        \includegraphics{by}
        \label{fig:by-sa}
    \end{flushright}
\end{figure}

\newpage

\section{Questions}

\par \emph{Notabilia} (\url{http://notabilia.net/}) is a project aimed at visualizing the 100 longest article for deletion discussions in the English Wikipedia.

\par Choose 3 of these discussions at random and answer the following questions:

\begin{itemize}
	\item In general, ¿do you think that participants follow standard netiquette guidelines?
	\item Can you find any specific guidelines in the Wikipedia project concerning this matter?
	\item Do you think it is easy to follow these discussions only by using indentation? Can you offer any suggestions to improve the interface to debate with other editors?
\end{itemize}

% section  (end)
\section{Answers}
\label{sec:answers}

\subsection{Articles selected}
\label{sub:list-of-articles}

\par I selected these three articles that represent each type in Notabilia classification:

\begin{itemize}
	\item \textbf{Controversial}: \url{http://en.wikipedia.org/wiki/Wikipedia:Articles_for_deletion/Zeitgeist_the_Movie}
	\item \textbf{Unanimous}: \url{http://en.wikipedia.org/wiki/Wikipedia:Articles_for_deletion/Israeli_Terrorism_against_the_United_States}
	\item \textbf{Swinging}: \url{http://en.wikipedia.org/wiki/Wikipedia:Articles_for_deletion/Republic_of_Ireland_vs_France_(2010_FIFA_World_Cup_Play-Off)}
\end{itemize}

\par In the three articles selected are displayed reviews for each 'vote'. The majority describes an argument supporting your choice. Respect among users is reflected in the discussions.
There are also issues that contribute only vote, with nothing to support it, that is, without argument. It is understood that not influence the final decision of the administrator.

\par As a curiously, I've noticed that in the article on Zeitgeist, abound entries marked as \emph{comment}. The arguments are very respectable between each person.

% subsection  (end)
\subsection{Wikiquette}
\label{sub:wikiquette}

\par Wikipedia has a netiquette guide, we can call it 'Wikiquette'. The article is called: \emph{Policies and guides}\footnote{\url{http://en.wikipedia.org/wiki/Wikipedia:Policies_and_guidelines}} Wikipedias behavioral guideline. Through this article, we stare to \emph{Polling is not a substitute for discussion}\footnote{\url{http://en.wikipedia.org/wiki/Wikipedia:Polling_is_not_a_substitute_for_discussion}}. 

\par Wikipedia, focuses on the consensus of the solution, not just a numerical result. So advocates consensus against the use of the polls. Because if you're really quem involved, make an argument for or against, shows an interest in the article. Unlike surveys, that only with a click, you get one vote, or otherwise, the solution may not be available for selection, so that participation could not provide a new argument. It would be a biased decision.

\par Deletion discussions\footnote{\url{http://en.wikipedia.org/wiki/Wikipedia:Deletion_discussions}} guide refers to our main question. Discussions by the removal, or not, of an article in Wikipedia.

\par This guide refers to our main question. Discussions by the removal, or not, of an article in Wikipedia. Provides the basis for proposing an article for deletion and guidelines for discussions deletion.

\par With a list of rules regarding Wikipedia respect, discussion and users:

\begin{itemize}
	\item Pages in user space
	\item Please do not take it personally
	\item Please be tolerant of others
	\item Sockpuppeting is not to be tolerated
	\item You may edit the article During the discussion
\end{itemize}

\par We offers a template to start the discussion on deleting or not an article\footnote{\url{http://en.wikipedia.org/wiki/Wikipedia:Templates_for_discussion}}.

\par It is above all, the following text in bold \textbf{should not be calculated solely by the balance of votes}.

\par By rough consensus administrators impartial judges must be the final decision on the item, which can result in three states:

\begin{itemize}
	\item keep
	\item delete
	\item no consensus
\end{itemize}

\par As a last point, we propose that if you still disagree with the result, you can re-propose the change in the selected item. Create a new article, but always having read the result of the above discussion and considering whether it can become a new eternal discussion without new arguments. Taking account of the process of reopening deleted article\footnote{\url{http://en.wikipedia.org/wiki/Wikipedia:Guide_to_deletion\#If_you_disagree_with_the_consensus}}.

% subsection  (end)

\subsection{Improvements}
\label{sub:improvements}

\par The focus on improving the visuals, as the main change. Notabilia provides a view from the distance counting the votes, positive and negative, over time. This view helps us to see the process from the outside but not to make decisions.

\par The important thing, as Wikipedia guide highlights, is the argument. Therefore it should facilitate access:

\begin{itemize}
	\item Reading the arguments of each voting over time.
	\item Divided by opinions, keep and delete.
	\item Divided by discussions.
\end{itemize}

\par In this way we obtain a more accurate comparison of opinions on the same side, watching the discussions in context and temporal evolution (though this part should be timeless, as mandated by the Wikipedia guide to writing an article). This could qualify better the opinions and monitor the discussions with more agility.

% subsection  (end)
% section  (end)
\begin{thebibliography}{9}

\bibitem{netiquette}
    RFC Netiquette Guidelines,\\
    \url{http://www.ietf.org/rfc/rfc1855.txt}

\end{thebibliography}

\end{document}
