\documentclass[11pt]{scrartcl}
\usepackage[parfill]{parskip}
\usepackage{graphicx}
\usepackage{booktabs}
\usepackage{tabulary}
\usepackage{float}
\usepackage{eurosym}
\usepackage{hyperref}

\graphicspath{{../images/}}

\title{\textbf{Project Management}}
\subtitle{Notabilia: Article for deletion}
\author{Ricardo Garc\'ia Fern\'andez}
\date{\today}

\begin{document}

\maketitle

\vfill

\begin{flushright}
    \copyright  2013 Ricardo Garc\'ia Fern\'andez - ricardogarfe [at] gmail [dot] com.

    This work is licensed under a Creative Commons Attribution- ShareAlike 3.0 License.
    To view a copy of this license visit:
 
    \url{http://creativecommons.org/licenses/by-sa/3.0/legalcode}.
\end{flushright}

\begin{figure}[h]
    \begin{flushright}	
        \includegraphics{by}
        \label{fig:by-sa}
    \end{flushright}
\end{figure}

\newpage

\section{Questions}

Notabilia (http://notabilia.net/) is a project aimed at visualizing the 100 longest article for deletion discussions in the English Wikipedia.

Choose 3 of these discussions at random and answer the following questions:

\begin{itemize}
	\item In general, ¿do you think that participants follow standard netiquette guidelines?
	\item Can you find any specific guidelines in the Wikipedia project concerning this matter?
	\item Do you think it is easy to follow these discussions only by using identation? Can you offer any suggestions to improve the interface to debate with other editors?
\end{itemize}


\end{document}
